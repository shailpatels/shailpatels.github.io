\documentclass[10pt,letterpaper]{article}
\usepackage[letterpaper,margin=0.5in]{geometry}
\setlength{\voffset}{-0.25in}
\usepackage[utf8]{inputenc}
\usepackage{mdwlist}
\usepackage[T1]{fontenc}
\usepackage{textcomp}
\usepackage{tgpagella}
\pagestyle{empty}
\setlength{\tabcolsep}{0em}

%add links to resume
\usepackage{hyperref}
\hypersetup{
    colorlinks=true,
    linkcolor=blue,
    filecolor=magenta,      
    urlcolor=blue,
}


\newenvironment{unindentsection}[1]%
{\begin{list}{}%
	{\setlength{\leftmargin}{-0.5#1}}%
	\item[]%
}
{\end{list}}

% format two pieces of text, one left aligned and one right aligned
\newcommand{\headerrow}[2]
{\begin{tabular*}{\linewidth}{l@{\extracolsep{\fill}}r}
	#1 &
	#2 \\
\end{tabular*}}

% make "C++" look pretty when used in text by touching up the plus signs
\newcommand{\CPP}
{C\nolinebreak[4]\hspace{-.05em}\raisebox{.22ex}{\footnotesize\bf ++}}

\begin{document}
\begin{center}
	{\LARGE \textbf{Shail Patel}}
	\vspace{0.2em}\\
    shailpatel67@gmail.com
	 \ $\mid$ \ \href{http://shailpatels.me/}{shailpatels.me} 
	 $\mid$ \href{https://github.com/shailpatels}{GitHub/shailpatels}\\
\end{center}

\subsection*{Education}
\begin{itemize}
	\parskip=0.1em
	\item 
	\headerrow
		{\textbf{Rensselaer Polytechnic Institute,} \emph{(RPI)}}
		{\emph{May 2021}}
	\\
	\headerrow
		{\emph{B.S. Computer Science, B.S. Cognitive Science}}
		{GPA 3.5/4.0}
\end{itemize}
\hrule
\vspace{-1em}
\subsection*{Core Technical Skills}
\textbf{Languages:} \  C, \ \CPP, \ C\#, \ PHP, \ Python, \ SQL, \ Java, \ JavaScript, \ TypeScript, \ HTML,\ CSS, \ Zig, \ Rust \vspace{0.5em}\\
\textbf{Software \& Tools:} Linux, \ git,\  CMake, \ CI/CD pipelines,\ Docker,\ VirtualBox,\ Ansible, \  \LaTeX \vspace{0.1in}\\
\textbf{Courses:}\, Intro to Algorithms, Data Structures, Computability \& Logic, Principles of Software, Operating Systems, Computer Communication Networks, Open Source Software, Computer Vision, Programming for Cognitive Science \& AI\\
\hrule

\subsection*{Work Experience}

\begin{itemize}
	\parskip=0.1em
	\item
	\headerrow
		{\textbf{Lockheed Martin Rotary \& Mission Systems}}
		{}
	\\
	\headerrow
		{\emph{Software Engineer, Level 2}}
		{\emph{June 2021 -- current}}
	\begin{itemize*}
		\item Engaged in modern \CPP \ development on a distributed system	
		\item Participated in an agile software engineering workflow 
        \item Developed and integrated GPU-accelerated signal processing chains through Nvidia CUDA toolkit 
        \item Designed and implemented a modular software architecture based on requirements, enabling rapid integration 
        \item Modified and integrated existing software modules to align with new specifications
        \item Collected statistics on performance and resource utilization to create formal documentation used to meet customer set requirements
        \item Communicated with program leadership and customer teams on the technical status of various subsystems
        \item Assisted multiple test events through the setup, deployment, and collection of results on site and in the field  
	\end{itemize*}
	\item
	\headerrow
		{\textbf{Submitty}} {\href{https://submitty.org/}{submitty.org}}
	\\
	\headerrow
		{\emph{Open Source Software Developer}}
		{\emph{Spring 2018 -- current}}
	\begin{itemize*}
		\item Full stack development in a Linux, Apache, PostgreSQL, and PHP/python 
			tech stack
		\item Implemented a QR code scanner to detect student IDs, automating the
			batching of student exams and saving hours of manual
			data entry for instructors 
		\item Designed and developed an OCR system to detect handwritten student ids
        \item Volunteered as a mentor for the organization during Google Summer of Code (GSoC) from 2019 to 2021
	\end{itemize*}
	\item
	\headerrow
		{\textbf{Cisco Systems}}
		{}
	\\
	\headerrow
		{\emph{Technical Consulting Engineer}}
		{\emph{January 2020 -- June 2020}}
	\begin{itemize*}
		\item Specialized in computer network engineering with a focus on troubleshooting customer equipment and networks 
		\item Racked, installed, and setup Cisco devices to build network replicas for issue reproduction 
        \item Worked with Cisco device configurations and basic layer 2 \& 3 network protocols 
		\item Co-Managed a project to automate detection of common issues in Cisco WebEx by leading and teaching a group of new developers in python and automation tools
	\end{itemize*}

\end{itemize}

\hrule
\subsection*{Publications}
\begin{itemize}
	\item
		\headerrow
		{A Graphical System For Logical Reasoning Using Existential Graphs}
		{\emph{RPI Cognitive Science Dept.}}
		\headerrow
		{\emph{Shail Patel, Bram van Heuveln}}
		{}
	\item
		\headerrow
		{Facilitating Discussion-Based Grading and Private Channels via an Integrated Forum}
		{\emph {SIGCSE 2019}}
		\headerrow
		{\emph {Andrew Aikens, Gagan Kumar, Shail Patel, Evan Maicus, Matthew Peveler, and Barbara Cutler}}
		{}
\end{itemize}
\vspace{-0.1em}
% \hrule
% \subsection*{Personal Projects}
% \begin{itemize}
% 	\item 
% 	\headerrow
% 		{{\bf The OpenBook project}} 
% 		{\href {https://github.com/openbook-project}{GitHub/OpenBook}}
% 		\headerrow
% 		{\emph{A replacement for traditional textbooks and ebooks}}
% 		{\emph{Spring 2019 -- current}}
% 	\begin{itemize*}
% 		\item Built an interpreter to read a markdown-like syntax to generate
% 			digital books in the form of HTML pages
% 		\item HackRPI 2019 Wolfram Alpha award, 15th place overall
% 	\end{itemize*}
% \end{itemize}
\vspace{-0.1em}
\hrule
\vspace{-0.25em}
\subsection*{Honors \& Certifications}
\begin{itemize}
    \item 
        \headerrow
        {Cisco Certified DevNet Associate \emph{(CCDevA)}}
	    {\emph{May 2023 -- May 2026}}
	    \headerrow
	    {Demonstrated knowledge over DevOps, network automation, and API design and development}
	    {}
    \item 
        \headerrow
	    {Cisco Certified Networking Associate \emph{(CCNA)}}
	    {\emph{May 2020 -- May 2026}}
	    \headerrow
	    {Demonstrated knowledge and skills related to network fundamentals}
	    {}
\end{itemize}
\end{document}
